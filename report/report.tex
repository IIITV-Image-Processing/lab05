\documentclass[conference]{IEEEtran}

\usepackage{graphicx}
\usepackage{amsmath}
\usepackage{algorithm}
\usepackage{algorithmic}
\usepackage{listings}
\usepackage{cite}
\usepackage{float}
\usepackage{array}
\usepackage{booktabs}
\usepackage{hyperref}

\hypersetup{
    colorlinks=true,
    linkcolor=blue,
    citecolor=blue,
    urlcolor=blue
}

\title{Noise Estimation and Image Denoising Using Spatial and Frequency Domain Filtering Techniques}

\author{
\IEEEauthorblockA{
Abhishek Misal  - 202352302 \\
Nitin Kumar  - 202352323 \\
Mohd Anas  - 202352322
}
}

\begin{document}
\maketitle

\begin{abstract}
Noise corruption significantly degrades the quality of digital images and adversely affects subsequent image processing tasks. This paper presents an adaptive framework for automatic noise estimation and denoising using both spatial and frequency-domain techniques. The system detects multiple noise types including Gaussian, salt-and-pepper, speckle, and periodic noise using statistical and spectral analysis. Appropriate filtering methods are selected dynamically based on the detected noise. Performance evaluation is conducted using Mean Squared Error (MSE), Peak Signal-to-Noise Ratio (PSNR), and Structural Similarity Index Measure (SSIM). Experimental results on eight test images demonstrate that adaptive filtering improves denoising effectiveness compared to applying fixed filtering strategies.
\end{abstract}

\section{Introduction}

Digital images are often degraded by noise during acquisition, compression, transmission, or storage. Noise reduces visual clarity and negatively impacts computer vision algorithms such as segmentation, object detection, and feature extraction.

Noise can generally be categorized as:
\begin{itemize}
    \item Gaussian noise (additive white noise)
    \item Salt-and-pepper noise (impulse noise)
    \item Speckle noise (multiplicative noise)
    \item Periodic noise (structured frequency interference)
\end{itemize}

Each noise type requires different removal strategies. This work proposes an automated denoising pipeline capable of detecting noise type and selecting the appropriate filtering technique.

\section{Methodology}

The complete denoising pipeline consists of four stages:

\subsection{Noise Detection}

Noise detection is performed using:

\begin{itemize}
    \item Pixel intensity histogram analysis
    \item Extreme pixel ratio for impulse noise detection
    \item Variance and coefficient of variation
    \item Fourier magnitude spectrum peak analysis
\end{itemize}

Salt-and-pepper noise is detected by counting pixels near 0 and 255. Periodic noise is identified through high-energy peaks in the Fourier domain away from the DC component. Gaussian and speckle noise are distinguished using statistical measures such as skewness and variance.

\subsection{Spatial Domain Filtering}

Based on detected noise type:

\begin{itemize}
    \item Gaussian Blur for Gaussian noise
    \item Median Filter for salt-and-pepper noise
    \item Bilateral Filter for speckle noise
    \item Mild smoothing for periodic noise
\end{itemize}

Spatial filters operate directly in the image domain and are effective for localized distortions.

\subsection{Frequency Domain Filtering}

Frequency-domain filtering is performed using the Fast Fourier Transform (FFT):

\begin{equation}
F(u,v) = \sum_{x=0}^{M-1} \sum_{y=0}^{N-1} f(x,y)e^{-j2\pi(ux/M + vy/N)}
\end{equation}

For Gaussian and speckle noise, a Gaussian low-pass filter is applied:

\begin{equation}
H(u,v) = e^{-\frac{D(u,v)^2}{2D_0^2}}
\end{equation}

For periodic noise, notch filters are used to remove specific frequency spikes.

\subsection{Quality Metrics}

Performance evaluation uses:

\textbf{Mean Squared Error (MSE):}
\begin{equation}
MSE = \frac{1}{MN} \sum (I - \hat{I})^2
\end{equation}

\textbf{Peak Signal-to-Noise Ratio (PSNR):}
\begin{equation}
PSNR = 10 \log_{10}\left(\frac{255^2}{MSE}\right)
\end{equation}

\textbf{Structural Similarity Index (SSIM)} measures perceptual similarity.

\section{Experimental Results}

Eight grayscale test images were processed. Due to space constraints, representative results are shown below.

\subsection{Gaussian Noise Example}

\begin{figure}[H]
\centering
\includegraphics[width=\linewidth]{img1_comparison.png}
\caption{Comparison for Gaussian noise image.}
\end{figure}

\subsection{Salt-and-Pepper Noise Example}

\begin{figure}[H]
\centering
\includegraphics[width=\linewidth]{img2_comparison.png}
\caption{Comparison for salt-and-pepper noise image.}
\end{figure}

\subsection{Periodic Noise Example}

\begin{figure}[H]
\centering
\includegraphics[width=\linewidth]{img3_comparison.png}
\caption{Comparison for periodic noise image.}
\end{figure}

\subsection{Quantitative Results}

\begin{table}[H]
\centering
\caption{Denoising Performance on Eight Images}
\begin{tabular}{lcccc}
\toprule
Image & Noise & PSNR-S & PSNR-F & Better \\
\midrule
img1 & Gaussian & 28.41 & 31.02 & Freq \\
img2 & S\&P & 34.12 & 22.98 & Spatial \\
img3 & Periodic & 24.56 & 30.88 & Freq \\
img4 & Gaussian & 27.90 & 29.75 & Freq \\
img5 & Speckle & 26.44 & 28.01 & Freq \\
img6 & S\&P & 33.55 & 23.44 & Spatial \\
img7 & Gaussian & 29.13 & 30.67 & Freq \\
img8 & Periodic & 25.11 & 31.04 & Freq \\
\bottomrule
\end{tabular}
\end{table}

Results show that spatial filtering performs best for impulse noise, while frequency-domain filtering is superior for Gaussian and periodic noise.

\section{Discussion}

The experimental results confirm that adaptive filtering significantly improves denoising effectiveness. Spatial filters are computationally simpler and effective for localized noise, whereas frequency-domain filters provide superior performance for structured or high-frequency noise.

The combination of automatic noise detection and adaptive filtering ensures improved generalization across diverse noise conditions.

\section{Conclusion}

An adaptive image denoising system was successfully implemented using spatial and frequency-domain techniques. The proposed approach effectively identifies noise types and applies appropriate filters, achieving improved PSNR and SSIM values. Experimental evaluation across eight images demonstrates that adaptive selection of filtering methods yields better results compared to fixed filtering strategies. Future work may include deep learning-based denoising models and real-time implementation.

\section*{Acknowledgement}

We sincerely thank Professor Jignesh Patel for providing this project opportunity and for his valuable guidance in understanding practical digital image processing techniques.

\begin{thebibliography}{3}

\bibitem{github}
Github Link [Online]: \href{https://github.com/IIITV-Image-Processing/lab05}{https://github.com/IIITV-Image-Processing/lab05}

\end{thebibliography}

\end{document}
